\documentclass[letterpaper,10pt]{article}

\usepackage{titling}
\usepackage{listings}
\usepackage{url}
\usepackage{setspace}
\usepackage{subfig}
\usepackage{sectsty}
\usepackage{pdfpages}
\usepackage{colortbl}
\usepackage{multirow}
\usepackage{relsize}
\usepackage{amsmath}
\usepackage{fancyvrb}
\usepackage{amsmath,amssymb,amsthm,graphicx,xspace}
\usepackage[titlenotnumbered,noend,noline]{algorithm2e}
\usepackage[compact]{titlesec}
\usepackage{paratype} 
\usepackage[T1]{fontenc}
\usepackage{tikz}
\usetikzlibrary{arrows,automata,shapes,trees,matrix,chains,scopes,positioning,calc}
\tikzstyle{block} = [rectangle, draw, fill=blue!20, 
    text width=2.5em, text centered, rounded corners, minimum height=2em]
\tikzstyle{bw} = [rectangle, draw, fill=blue!20, 
    text width=4em, text centered, rounded corners, minimum height=2em]

\definecolor{namerow}{cmyk}{.40,.40,.40,.40}
\definecolor{namecol}{cmyk}{.40,.40,.40,.40}

\let\LaTeXtitle\title
\renewcommand{\title}[1]{\LaTeXtitle{\textsf{#1}}}


\newcommand{\handout}[5]{
  \noindent
  \begin{center}
  \framebox{
    \vbox{
      \hbox to 5.78in { {\bf ECE254: Operating Systems and Systems Programming } \hfill #2 }
      \vspace{4mm}
      \hbox to 5.78in { {\Large \hfill #4  \hfill} }
      \vspace{2mm}
      \hbox to 5.78in { {\em #3 \hfill} }
    }
  }
  \end{center}
  \vspace*{4mm}
}

\newcommand{\lecture}[3]{\handout{#1}{#2}{#3}{Lecture #1}}
\newcommand{\tuple}[1]{\ensuremath{\left\langle #1 \right\rangle}\xspace}

\addtolength{\oddsidemargin}{-1.000in}
\addtolength{\evensidemargin}{-0.500in}
\addtolength{\textwidth}{2.0in}
\addtolength{\topmargin}{-1.000in}
\addtolength{\textheight}{1.75in}
\addtolength{\parskip}{\baselineskip}
\setlength{\parindent}{0in}
\renewcommand{\baselinestretch}{1.5}
\newcommand{\term}{Spring 2015}

\singlespace


\begin{document}

\lecture{ 32 --- File System Interface }{\term}{Jeff Zarnett}

\section*{File Systems}

The file system is an important and highly visible part of the operating system. More than just the way of storing data and programs, persistently, it also provides organization for the files through a directory structure and maintains metadata related to files. 

But what is a file? The snarky UNIX answer is, ``Everything is a file!'', but using the word in the definition is rather bad form. As far as the computer is concerned, any data is just 1s and 0s (bytes). The file is just a logical unit to organize these. So an area of disk is designated as belonging to a file. Files can contain programs (e.g., \texttt{word.exe}) and/or data (e.g., \texttt{book-report.doc}). The content of a file is defined by its creator. The creator could be a user if he or she is using notepad or something, or it could be a program, like a compiler creating an output binary file.

Files typically have attributes, which, although they can vary, tend generally to include the following things~\cite{osc}:

\begin{enumerate}
	\item \textbf{Name}: The symbolic file name, in human-readable form.
	\item \textbf{Identifier}: The unique identifier, usually a number, that identifies the file inside the file system.
	\item \textbf{Type}: Information about what kind of file it is.
	\item \textbf{Location}: The physical location of the file, including what device (e.g., hard drive) it is on.
	\item \textbf{Size}: The current, and possibly maximum, size of the file.
	\item \textbf{Protection}: Access-control information, including who owns the file, who may read, write, and execute it...
	\item \textbf{Time, Date, User ID}: The owner of the file, time of creation, last access, last change... any sort of data that is useful for protection, security, usage monitoring...
\end{enumerate}

Files are maintained in a directory structure. The directory structure is generally quite familiar to us as the folders on the system. Directories, really, are just like files; they are information about what files are in what locations, and they too will be stored on disk.

\subsection*{File Operations}
It makes some sense to consider a file to be something like a class in an object-oriented programming language; a file has some data (fields, metadata) and some operations. The OS provides these operations to allow users to work with and on files. Six basic operations are required for a file system to be useful, though other things like renaming and so on are nice to have~\cite{osc}:

\begin{enumerate}
	\item Creating a file.
	\item Writing a file.
	\item Reading a file.
	\item Repositioning within a file.
	\item Deleting a file.
	\item Truncating a file.
\end{enumerate}

Let's examine each of these briefly. As you will see, there is a certain similarity between file operations and memory operations, with which we should already be familiar.

\paragraph{Creating a File.} Like allocating memory, creating a new file has multiple essential steps: first, find a place to put the file, allocate that space and mark that file as being allocated, and finally put the file in its appropriate directory.

\paragraph{Writing a File.} Writing a file requires the name or identifier of the file and the data to be written to the file. Using the name or identifier, the system finds that file and can then start putting data in the file. A write operation may replace the existing contents or append (write at the end) to the existing contents. A pointer will be needed to keep track of where the next write will take place, and will be updated after each write.

\paragraph{Reading a File.} This requires the name or identifier of the file and where in memory the next block of the file should be put. A pointer will also be required to indicate where the next read will take place. 

\paragraph{Repositioning within a File.} Since a file may be read or written, but usually only one at a time, the pointer for the write location may in fact be the same pointer for the memory location. If so, we might call it a current position pointer, and this operation is just repositioning it within the file. Repositioning is also sometimes called a seek operation.

\paragraph{Deleting a File.} Deletion works pretty much as we would expect: find the file, mark its space as free, and remove it from the directory listing. This is a ``simple'' deletion and it does not actually get rid of any of the data, it just makes the file system forget the existence of the file. However, it might be possible to recover the data if the space it previously occupied has not been overwritten. This is a bit like a freed pointer in C possibly still being accessible. Some systems offer a more secure deletion routine that overwrites the space the file used to occupy with zeros.

\paragraph{Truncating a File.} If a file should be erased but its attributes maintained (e.g., all the metadata), we can truncate it: cut off all the contents. The file length is reset to zero and its data area is marked as free, but the rest of the attributes remain the same.


These six operations can be combined for most of the other things we may want to do. To copy a file, for example, create a new file, read from the old file, and write it into the new file. We may also have operations to allow a user to access or set various attributes such as the owner, security descriptors, size on disk, et cetera~\cite{osc}.

Aside from creation and deletion, all the other operations are restricted to files that are open. When a file is opened, a program gets a reference to it, and the operating system keeps track of which files are currently open in which process. It is good behaviour for a process to close a file when it is no longer using it, but eventually when the process terminates, that will automatically close any open files (hopefully).

The implementation of the open and close file routines can be very complicated: we might open a file as read-only (modifications forbidden) etc. If the opening mode allows multiple processes to open the file simultaneously, the OS will likely maintain a per-process table and a system-wide table. Access rights and accounting information may appear in either. The per-process table contains a reference into the system-wide table. When the last process that has a file open uses the close system call, the reference count is zero and we can remove it from the table.

Some operating systems support file locks that work a lot like the mutual exclusion concepts we have examined earlier. Locks may be exclusive, or non-exclusive. When a file is locked by one process, other processes will be advised that opening failed due to someone having a lock on that file. Similarly, files in use cannot be deleted while that file is in use. 

Windows, for example, uses locking and any file that is open in some program cannot be deleted. UNIX, however, does not, so UNIX-compatible programs can, if they need, lock a file, but by default this does not happen. In UNIX if a file is open in a program, another user can still delete the file and it will be removed from the directory. As long as that program remains open and retains that reference to the file, it can still operate on that file. However, once the file is no longer open in a program, its storage space will be marked as free.

\subsection*{File Types}

Files we are familiar with often have extensions separated from the file name by a period, like \texttt{fork.txt}. The \texttt{txt} extension tells us some information about the file, i.e. that it is a text file. These things are mostly hints to the OS or user about what sort of file it is. In most operating systems, any program can open arbitrary files... that it has a \texttt{.docx} extension is only a suggestion that it should be opened by a word processing program, but nothing stops people from opening it in any other program. OSes typically allow setting a default program for the extension: e.g., always open \texttt{.docx} files with LibreOffice.

\subsection*{Access Methods}

When the system wants to operate on a file, it is usually read into main memory. We often have several options if we are dealing with a hard disk drive where we can read from or write to anywhere. But this is not the only way.

\textit{Sequential access} is the simplest method. Information in the file is processed in order, one section after another. Compilers tend to operate in this way, for example. Reads and writes make up most of the operations; the operations may look something like \texttt{read\_next} which advances and reads from the next record. It is possible that going backwards is allowed, either by resetting to the beginning or ``reversing'' by some number of records (possibly 1). This access method stems from tape drive systems. We could advance or reverse the tape one section at a time (and it was not THAT long ago that tape drives were used for large data backups)~\cite{osc}.

The normal kind of access we are used to in dealing with files is called \textit{direct access}. It is exactly what you expect: any part of a file can be accessed at any time. There are no restrictions on the order and no need for the program to seek around to the right place. Rather than reading the next block or advancing to the next block, we can usually read from or seek to an arbitrary block. 

The operating system typically hides the underlying physical block information by using a relative block number. From the perspective of the application or user, the first block of the file is numbered zero, the next one is one, and so on. This may, however, have no underlying relation to where a block is located on disk.

Locating an offset inside a file can be a pain for an operating system. Disk systems operate on blocks of equal size, but the logical size of the file probably does not fit exactly into a block (or an integer multiple of blocks). So we often have multiple logical records packed into one block. As far as UNIX is concerned, any byte of a file can be accessed, but if the disk uses 4~KB blocks, the file system will need to pack bytes into and out of the physical blocks to make it work~\cite{osc}.

Because disk space is allocated in a block, there is some internal fragmentation inherent to the system. If a file is, say, 21~KB, it will fit in 5 of the 4~KB blocks and a fifth block will be allocated that contains the last 1~KB but has 3~KB of internal fragmentation (wasted space).

\subsection*{Directories}

A directory is really just a symbol table that translates file names (user-readable representations) to their directory entries. A directory should support several common operations~\cite{osc}:

\begin{enumerate}
	\item \textbf{Search}. We want to be able to find a file, and searching is typically not just on the file name but may include the contents of files as well, if their content is human-readable data.
	\item \textbf{Add a File}. Add a file to the directory.
	\item \textbf{Remove a File}. Remove a file from the directory.
	\item \textbf{List a Directory}. List the files of the directory and the contents of the directory entry.
	\item \textbf{Rename a File}. Change the user-friendly file name, possibly changing the file's position in the directory if it is sorted by name.
	\item \textbf{Navigate the File System}. It should be possible to open subdirectories, go to parent directories, and so on.
\end{enumerate}

There are some simple file systems where there are no such things as subdirectories, but they don't really require any examination. Textbooks may also bring up a structure where each user has his or her own directly but cannot have subdirectories either. Also rather uninteresting. The kind of directory we are most familiar with is tree-structured: there it a root directory, and every file in the system has a unique name when the name and path to it (from the root) are combined.

In UNIX the root directory is just called \texttt{/} (forward slash) and from there we can navigate to any file. If we would like to run the \texttt{ls} command, we will find it in the \texttt{bin} directory as \texttt{/bin/ls}. This is an example of an absolute path. Most of the time we do not have to use the absolute path (the full file name); a relative path (the path from the current directory) will suffice. As an example, if you want to compile something with a command like \texttt{gcc code/example.c}, the file \texttt{example.c} is in a subdirectory of the current directory called \texttt{code} and the system will work out that we need to start from the current directory (e.g., \texttt{/home/jz/ece254/}) and prepend that to the given file name, to produce the absolute path of \texttt{/home/jz/ece254/code/example.c}.

OS designers have to make a choice about deletion of directories, if a directory is not empty. If it is empty, just removing the directory is enough. If it contains some files, either the system can refuse to delete the directory until it is empty, or automatically delete the files and subdirectories in the system in their entirety. Also, what does it mean to delete a file or folder? In modern operating systems, the delete command sometimes does not necessarily actually delete the file or folder, but instead moves it to some deleted files directory (recycle bin, trash can, whatever you want to call it). If it is deleted from there then it is really gone, but while it is in that deleted file directory it can be restored.

File systems may also support the sharing of files: there is one copy of the file but it has more than one name. In UNIX this concept is called a \textit{link} and this is effectively a pointer to another file. Links are either ``hardlinks'' or ``symlinks''. 

Symlinks, or symbolic links, are just references by file name. So if a symbolic link is created to a file like \texttt{/Users/jz/file.txt}, the symbolic link will just be a ``shortcut'' to that file. If the file is later deleted, the symbolic link is left pointing to nothing. A future attempt to use this pointer will result in an error, so it is something to check on. It would be expensive, though possible, to search through the file system to find all links and remove them.

Creating a hardlink means creating a second pointer to the underlying file in the file system. If a hardlink exists and the user deletes that file, the file still remains on disk until the last hardlink is removed. We will see later on, when we examine the file system implementation, how this actually works, but the short answer is: reference counting. The file structure maintains a count of how many hardlinks reference a file, and it is only really deleted if the count falls to zero.

\subsection*{File Permissions}
To protect users from one another and to maintain the confidentiality and integrity of data, files usually have some permissions associated with them, which may control access to the following operations~\cite{osc}:

\begin{enumerate}
	\item Read
	\item Write
	\item Execute
	\item Append (write at the end of the file)
	\item Delete
	\item List (view the attributes of the file)
\end{enumerate} 

\paragraph{UNIX-Style Permissions.}
UNIX-Style permissions are commonly used still today in a lot of UNIX and UNIX-like systems. Each file has an owner and a group, and a set of permissions that can be assigned for the owner, the group, and for everyone. There are three basic permissions: read, write, and execute (run as a program). The permissions are represented using 10 bits, where a 1 indicates true and a 0 indicates false. The first bit is the directory bit and indicates if the file being examined is actually a directory. The next three bits are the read, write, and execute bits for the owner, followed by the read, write, and execute bits for the group, and finally the read, write, and execute bits for everyone.

Effective permissions are determined by the user: the owner of the file gets the owner permissions even if different permissions are assigned to the group or everyone. Precedence goes from left to right: owner takes precedence over the group; group takes precedence over the permissions for everyone.

The permissions can be shown to the screen in a human-readable format which is ten characters long. The order is always the same, and so a dash (\texttt{-}) appears if a bit is zero (permission does not exist). The character \texttt{d} is used to indicate a directory, \texttt{r} to indicate read access, \texttt{w} to indicate write access, and \texttt{x} to indicate execute access.

Example: permissions of \texttt{-rwxr{-}{-}{-}{-}{-}} indicate that a file is not a directory; the owner can read, write, and execute; other members of the group can read it only, and everyone else has no access to the file (cannot read, write, or execute). 

Permissions can also be written in octal (base 8) where r = 4, w = 2, and x = 1. To get the octal representation, start with 0, and then add the value of the  permissions that are present, using zero where permissions are absent. You might then have a permission that reads \texttt{750} - meaning that the owner has read, write and execute access (0 + 4 + 2 + 1 = 7); group members have read and execute access (0 + 4 + 0 + 1 = 5); everyone else has no access to it at all.

There are some more details like what the permissions mean on directories, and some advanced topics like \texttt{setuid}, \texttt{setgid}, and ``sticky bit'', but we will not cover them in this course.

The obvious shortcoming of this approach is that it is very coarse-grained: there are only three groups for whom we can specify permissions. Within Linux and other similar systems there is a trend now towards using SELinux (Security Enhanced Linux) which is an Access Control List system.

\paragraph{Access Control Lists.}
In SELinux and NTFS (Windows NT File System), files are protected by Access Control Lists. In such a system, each file can have as many security descriptors as we like, in which we specify the permissions a specific user or group is supposed to have. Similarly, the list of permissions need not be restricted to read-write-execute; NTFS, for example, has several different permissions such as ``Take Ownership''.

In NTFS, you can see the security descriptors for a file rather easily by right clicking it in Explorer, opening the properties dialog, and choosing the security tab. The descriptors have two checkboxes: allow and deny, with deny taking precedence over allow.

Permissions can start out at ``default deny'', in which case only the users explicitly granted access may access the file, or ``default permit'' in which case everyone has access, minus those users whose privileges are explicitly denied. Default deny is obviously more secure.

A simple way to represent an access control list is a set of tuples listing the users and permissions. For example:
\begin{verbatim}
	(alice, read)
	(bob, read)
	(charlie, read)
	(charlie, write)
	(bob, execute)
\end{verbatim}

Access Control Lists have a complexity that the UNIX file permissions do not: the idea of inheritance. We'll examine, briefly, inheritance in NTFS. Inheritance is supposed to be a convenience feature, but it can often be a cause of problems. If there is a directory with ACLs established and object inheritance is enabled, then any new file crated in this directory will receive the same ACL of the directory. The danger is that any existing file moved into the directory will retain its original ACL. If it is supposed to get the ACL of the directory, it needs to be explicitly set~\cite{ntfsacl}.

\bibliographystyle{alpha}
\bibliography{254}


\end{document}