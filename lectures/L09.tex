\documentclass[letterpaper,10pt]{article}

\usepackage{titling}
\usepackage{listings}
\usepackage{url}
\usepackage{setspace}
\usepackage{subfig}
\usepackage{sectsty}
\usepackage{pdfpages}
\usepackage{colortbl}
\usepackage{multirow}
\usepackage{relsize}
\usepackage{amsmath}
\usepackage{fancyvrb}
\usepackage{amsmath,amssymb,amsthm,graphicx,xspace}
\usepackage[titlenotnumbered,noend,noline]{algorithm2e}
\usepackage[compact]{titlesec}
\usepackage{paratype} 
\usepackage[T1]{fontenc}
\usepackage{tikz}
\usetikzlibrary{arrows,automata,shapes,trees,matrix,chains,scopes,positioning,calc}
\tikzstyle{block} = [rectangle, draw, fill=blue!20, 
    text width=2.5em, text centered, rounded corners, minimum height=2em]
\tikzstyle{bw} = [rectangle, draw, fill=blue!20, 
    text width=4em, text centered, rounded corners, minimum height=2em]

\definecolor{namerow}{cmyk}{.40,.40,.40,.40}
\definecolor{namecol}{cmyk}{.40,.40,.40,.40}

\let\LaTeXtitle\title
\renewcommand{\title}[1]{\LaTeXtitle{\textsf{#1}}}


\newcommand{\handout}[5]{
  \noindent
  \begin{center}
  \framebox{
    \vbox{
      \hbox to 5.78in { {\bf ECE254: Operating Systems and Systems Programming } \hfill #2 }
      \vspace{4mm}
      \hbox to 5.78in { {\Large \hfill #4  \hfill} }
      \vspace{2mm}
      \hbox to 5.78in { {\em #3 \hfill} }
    }
  }
  \end{center}
  \vspace*{4mm}
}

\newcommand{\lecture}[3]{\handout{#1}{#2}{#3}{Lecture #1}}
\newcommand{\tuple}[1]{\ensuremath{\left\langle #1 \right\rangle}\xspace}

\addtolength{\oddsidemargin}{-1.000in}
\addtolength{\evensidemargin}{-0.500in}
\addtolength{\textwidth}{2.0in}
\addtolength{\topmargin}{-1.000in}
\addtolength{\textheight}{1.75in}
\addtolength{\parskip}{\baselineskip}
\setlength{\parindent}{0in}
\renewcommand{\baselinestretch}{1.5}
\newcommand{\term}{Spring 2015}

\singlespace


\begin{document}

\lecture{ 9 --- POSIX Threads (the pthread)}{\term}{Jeff Zarnett}

\section*{POSIX Threads}

The term \texttt{pthread} refers to the POSIX standard (also known as the IEEE 1003.1c standard) that defines thread behaviour in UNIX and UNIX-like systems (Linux, Mac OS X, Solaris...). This is a specification document that says how threads should behave.This standard lets code for one UNIX-like system (e.g., Solaris) run easily on another (e.g., Linux). The POSIX standard for pthreads defines something like 100 function calls, but we need not examine all of them. Let us focus on a few of the important ones and we can see they are, for the most part, very similar to what we saw with parent and child processes~\cite{mos}:

\begin{itemize}
	\item \texttt{pthread\_create} -- Create a new thread. This works a lot like \texttt{fork}.
	\item \texttt{pthread\_exit} -- Terminate the calling thread. This is like \texttt{exit} in that it ends execution and returns a value.
	\item \texttt{pthread\_join} -- Wait for a specific thread to exit. This is like \texttt{wait}: the caller cannot proceed until the thread it is waiting for calls \texttt{pthread\_exit}. Note that it is an error to join a thread that has already been joined.
	\item \texttt{pthread\_yield} -- Release the CPU and let another thread run. There is no analogous call for processes\footnote{At least not in UNIX. In Mac OS 9, however...} because every process is expected to act as if it is the only process in the system. In the case of threads, however, since they all belong to the same program, we expect that threads want to co-operate rather then compete for CPU time and threads can make decisions about when it would be ideal to let some other thread run instead.
	\item \texttt{pthread\_attr\_init} -- Create and initialize a thread's attributes. The attributes contain things like the priority of the thread. (``After you, sir.'' ``Oh no, after you.'')
	\item \texttt{pthread\_attr\_destroy} -- Remove a thread's attributes. Free up the memory holding the thread's attributes. This does not terminate the threads.
	\item \texttt{pthread\_cancel} -- Signal cancellation to a thread; this can be asynchronous or deferred, depending on the thread's attributes.
\end{itemize}


So far we have examined threads at a theoretical level, but have not actually considered how to make something run in the background. Noting that \texttt{main} is just a function (with a special name), the way we can start a thread is to say: run this function, but in a new thread. The system call to create a new thread is \texttt{pthread\_create} and its use looks like~\cite{mte241}:

\begin{verbatim}
pthread_create( pthread_t *thread,
                const pthread_attr_t *attributes,
                void *(*start_routine)( void * ),
                void *argument );
\end{verbatim}

Where: \texttt{thread} is a pointer to a \texttt{pthread} identifier and will be assigned a value when the thread is created. The attributes \texttt{attr} may contain various characteristics (but you may supply \texttt{NULL} if you want the defaults). The \texttt{start\_routine} is any function that takes a single untyped pointer and returns an untyped pointer. Finally, the last parameter, \texttt{arguments} is the argument passed to the \texttt{start\_routine}.

After the new thread has been created, the process has two threads in it. The OS makes no guarantee about which thread will be executing after the new one is created; this is a matter of scheduling. It could be either of the threads of the process, or a different process entirely.

Our experience with C-like languages suggests it is normal to have a single return value from a function, but it seems limiting to be able to put in just one parameter. There are two ways to get around this: with an array or with structures (\texttt{struct}). In the case of the array, the argument provided to \texttt{pthread\_create} is just a pointer to the array. This is also, incidentally, how you can get multiple return values out of a function in Java or C\# (\texttt{public Object[] foo()}), but I don't recommend it as a good programming practice. The other way to do it is to use the \texttt{struct} as below~\cite{mte241}, defining a structure for the parameter type and one for the return type. In the example, all four variables are integers, but they could be of any type.

\begin{verbatim}
typedef struct {
  int parameter1;
  int parameter2;
} parameters_t;

typedef struct {
  int return1;
  int return2;
} return_t;
\end{verbatim}

The function that is to run in the new thread must expect a pointer to the arguments rather than explicit arguments:\\
\texttt{void* function( void *args )}\\
which can then be cast to the appropriate type:\\
\texttt{parameters\_t *arguments = (parameters\_t*) args;}


What about the thread attributes? They can be used to query or set specific attributes of the thread, such as~\cite{pthreads}:
\begin{itemize}
	\item Detached or joinable state
	\item Scheduling data (policy, parameters, etc...)
	\item Stack size
	\item Stack address
\end{itemize}

The first item in that list indicates if a thread is joinable or detached. By default, new threads are usually joinable (that is to say, that some other thread can call \texttt{pthread\_join} on them). As noted before, it is a logical error to attempt multiple joins on the same thread. To prevent a thread from ever being joined, it can be created in the detached state (or the method \texttt{pthread\_detach} can be called on a joinable thread). Trying to join a detached thread is also a logical error~\cite{pthreads}.

The use of \texttt{pthread\_exit} is not the only way that a thread may be terminated. Sometimes we want the thread to persist (hang around), but if we want to get a return value from the thread, then we need it to exit. Like the \texttt{wait} system call, the \texttt{pthread\_join} is how we get a value out of the spawned thread~\cite{mte241}:

\begin{verbatim}
pthread_create( &thread_id, NULL, function_name, &args );

// This function and the created function are now running in parallel 
void *void_ret;

pthread_join( thread_id, &void_ret );

return_t *returnValue = (return_t *) void_void_argret;
\end{verbatim}

If a thread has no return values, it can just \texttt{return NULL;} which will have the same effect as \texttt{pthread\_exit} and send \texttt{NULL} back to the thread that has joined it. If the function that is called as a task returns normally rather than calling the exit routine, the thread will still be terminated. 

Another way a thread might terminate is if the \texttt{pthread\_cancel} function is called with it as the target. As before, if the termination is deferred rather than asynchronous, the thread is responsible for cleaning up after itself before it stops.

A thread may also be terminated indirectly: if the entire process is terminated or if \texttt{main} finishes first (without calling \texttt{pthread\_exit} itself). Indeed, \texttt{main} can use \texttt{pthread\_exit} as the last thing that it does. Without that, \texttt{main} will not wait for other, unjoined threads to finish and they will all get suddenly terminated. If \texttt{main} calls \texttt{pthread\_exit} then it will be blocked until the threads it has spawned have finished~\cite{pthreads}.



Let us examine a slightly more complex example that invokes more of the pthread system calls. The code sample below from~\cite{osc} provides an example of a multithreaded C program that uses pthreads to calculation the summation of a non-negative integer in a second thread. 

\begin{verbatim}
#include <pthread.h>
#include <stdio.h>

int sum; /* this data is shared by the thread(s) */
void *runner(void *param); /* threads call this function */
int main(int argc, char *argv[]) {

  pthread_t ti; /* the thread identifier */
  pthread_attr_t attr; /* set of thread attributes */

  if (argc != 2) {
    fprintf(stderr,"usage: a.out <integer value>\n"); 
    return -1;
  }
  if (atoi(argv[1]) < 0) {
    fprintf(stderr, "%d must be >= 0\n", atoi(argv[1])); 
    return -1;
  }

  /* get the default attributes */
  pthread_attr_init(&attr);
  /* create the thread */
  pthread_create(&tid, &attr, runner, argv[1]); /* wait for the thread to exit */
  pthread_join(tid, NULL); 
  printf("sum = %d\n", sum);
}

/* The thread will begin control in this function */ 
void *runner(void *param) {

  int i, upper = atoi(param);
  sum = 0;
  for (i = 1; i <= upper; i++) {
    sum += i;
  }
  
  pthread_exit(0);
}
\end{verbatim}

In this example, both threads are sharing the global variable \texttt{sum}. We have some form of co-ordination here because the parent thread will join the newly-spawned thread (i.e., wait until it is finished) before it tries to print out the value. If it did not join the spawned thread, the parent thread would print out the sum early. This is yet another example of that subject that keeps popping up: co-ordination...


Fork and exec


Signal handling

More about cancellation


\bibliographystyle{alpha}
\bibliography{254}


\end{document}